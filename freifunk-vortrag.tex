\documentclass[aspectratio=1610]{beamer}

\usepackage[utf8]{inputenc}
\usepackage[ngerman]{babel}
\usepackage[T1]{fontenc}
\usepackage[absolute,overlay]{textpos}
\usepackage{xcolor}

\definecolor{black}{HTML}{1C1C1C}

\mode<presentation>
{
  \usetheme{Madrid}
}

\newcommand{\boxx}[1]{
  {#1}
}

\newcommand{\photoby}[1]{
  \begin{textblock*}{\paperwidth}[0.2,3.5](\textwidth,\textheight)
    \raggedright{
      \tiny{
        \colorbox{black}{
        \color{white}{#1}
      }
      }
    }
  \end{textblock*}
}

\newcommand{\photo}[3]{
  {
    \usebackgroundtemplate{
      \includegraphics[width=\paperwidth]{#2}
    }
    \begin{frame}{\boxx{#1}}
      \photoby{#3}
    \end{frame}
  }
}

\title[Warum wir Freifunk machen]{Warum wir Freifunk machen}
\author[nomaster]{Mic \flq nomaster@chaosdorf.de\frq}
\institute{Freifunk Düsseldorf}
\date[SIGINT13]{SIGINT 2013}

\begin{document}

\huge

\begin{frame}
  \titlepage
\end{frame}


\begin{frame}{Weil wir Nerds sind.}
  \begin{itemize}
    \pause
    \item nicht alle, aber viele von uns
    \pause
    \item und wir lieben es, Infrastruktur aufzubauen
    \pause
    \item ja, auch nicht alle
  \end{itemize}
\end{frame}

\begin{frame}{Freifunk is not dead.}
  \begin{itemize}
      \pause
      \item Aktiv seit 10 Jahren
      \pause
      \item Skaliert immer noch nicht
      \pause
      \item Dafür B.A.T.M.A.N.
      \pause
      \item und wächst immer weiter
  \end{itemize}
\end{frame}

\photo{Open Wifi Map \tiny Freifunk Rheinland}{img/map.jpg}{map.freifunk-rheinland.net}

\begin{frame}{Wir kommen in Frieden.}
  \begin{itemize}
    \pause
    \item Hackerspace ist Einsatzzentrale
    \pause
    \item Nachbarn ans Netz bekommen
    \pause
    \item Zusammenarbeit mit anderen Kollektiven
    \pause
    \item Nerds sind überall
  \end{itemize}
\end{frame}

\begin{frame}{Bring’ Freifunk mit!}
  \begin{itemize}
    \pause
    \item zum Imbiss
    \item zum Frisör
    \item ...
  \end{itemize}
\end{frame}

\photo{Wolke beim Niemandsland e.V.}{img/niemandsland-map.png}{freifunk-rheinland.net}
\photo{Den Häuserblock versorgen}{img/node20.png}{freifunk-rheinland.net}

\begin{frame}{Das größte Problem ist die Angst vor Repression.}
  \begin{itemize}
    \pause
    \item Uplinks per VPN Tunnel
    \pause
    \item hat den Zuspruch immens vergrößert
    \pause
    \item Doppelstrategie mit Klagen gegen Störerhaft
  \end{itemize}
\end{frame}

\photo{Autonomes Zentrum Köln}{img/azkoeln.jpg}{strassenstriche.net}

\begin{frame}{Kommunikation ist alles.}
  \begin{itemize}
    \pause
    \item Regelmäßige Treffen
    \pause
    \item Workshops, Hackathons, Meetings
    \pause
    \item Freifunk ist ein Lernprojekt
  \end{itemize}
\end{frame}

\photo{Coworking Space GarageBilk}{img/garagebilk.jpg}{Violeta Pelivan}

\begin{frame}{Strahlung gibt es immer und überall.}
  \begin{itemize}
    \pause
    \item Aufklärung betreiben
    \pause
    \item WLAN ist relativ strahlungsarm
    \pause
    \item in Städten völlig überbuchte Kanäle
    \pause
    \item Funkverschmutzung reduzieren
  \end{itemize}
  \pause
  Wo sind die Aluhüte?
\end{frame}

\photo{Niemandsland Düsseldorf}{img/niemandsland.jpg}{Niemandsland e.V.}

\begin{frame}{Erfolge}
  \begin{itemize}
    \pause
    \item Öffentliche Orte mit Freifunk ausgestattet
    \pause
    \item Empowerment für selbstbezeichnete Noobs
    \pause
    \item Operation Störerhaft finanziert und gestartet
  \end{itemize}
\end{frame}

\photo{Operation Störerhaft}{img/operation.png}{freifunk-rheinland.net}

\begin{frame}{Message}
  \Huge\center
  Support your local Freifunk cloud!
\end{frame}

\end{document}
